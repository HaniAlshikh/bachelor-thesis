% !TEX root = ./thesis.tex
% List of all symbols, acronyms and glossary entries
% @author Thomas Lehmann
%

% Warning: Lable must only be used once

% Glossary entries
\newglossaryentry{gl:haw}{
  name={HAW Hamburg},
  description={Die HAW Hamburg ist die vormalige Fachhochschule am Berliner Tor}
  }

\newglossaryentry{gl:go}{
  name={GO},
  description={\todo{TODO}}
  }

\newglossaryentry{gl:ef}{
  name={EventFormatter},
  description={An event formatter is responsible for creating a human-readable representation of a given event},
  plural={EventFormatters}
  }

\newglossaryentry{gl:ei}{
  name={emissary ingress},
  description={\todo{TODO}},
  }

\newglossaryentry{gl:envoy}{
  name={envoy},
  description={\todo{TODO}},
  }

\newglossaryentry{gl:ac}{
  name={Audit Component},
  description={\todo{TODO}},
  }

\newglossaryentry{gl:ab}{
  name={Audit Browser},
  description={\todo{TODO}},
  }

\newglossaryentry{gl:docker}{
  name={Docker},
  description={\todo{TODO}},
  }

\newglossaryentry{gl:make}{
  name={Make},
  description={\todo{TODO}},
  }


\newglossaryentry{gl:helm}{
  name={Helm},
  description={\todo{TODO}},
  }

\newglossaryentry{gl:kubectl}{
  name={Kubectl},
  description={\todo{TODO}},
  }

\newglossaryentry{gl:kind}{
  name={Kind},
  description={\todo{TODO}},
  }

\newglossaryentry{gl:step}{
  name={Step-cli},
  description={\todo{TODO}},
  }

\newglossaryentry{gl:idpt}{
  name={idempotent},
  description={Idempotency is a property of a system or operation where the result of performing that operation multiple times is the same as performing it once. In other words, if an idempotent operation is performed multiple times, the end result is the same as performing it only once.},
  }

\newglossaryentry{gl:vc}{
  name={Version Control},
  description={},
  }

\newglossaryentry{gl:est}{
  name={EventStore},
  description={\todo{TODO}}
  }

\newglossaryentry{gl:monogui}{
  name={MonoGUI},
  description={is a \acrlong{ac:gui} for \glsfirst{ac:m8}}
  }

\newglossaryentry{gl:12faktor}{
  name={The Twelve-Faktor App},
  description={}
  }

% Hybrid
\newglsacronym{ci}{CI}{Continuous Integration}{\todo{TODO}}
\newglsacronym{cd}{CD}{Continuous Delivery}{\todo{TODO}}
\newglsacronym{k8s}{k8s}{Kubernetes}{
  \begin{quote}
  also known as K8s, is an open-source system for automating deployment, scaling, and management of containerized applications.
  \end{quote}\citep{Kubernetes}
  }
\newglsacronym{crd}{CRD}{Custom Resource Definition}{\todo{TODO}}
\newglsacronym{pki}{PKI}{Public Key Infrastructure}{\todo{TODO}}
\newglsacronym{es}{ES}{Event-Sourcing}{\todo{TODO}}
\newglsacronym{m8}{m8}{Monoskope}{(short m8 spelled "mate") implements the management and operation of tenants, users and their roles in a Kubernetes multi-cloud multi-cluster environment.}

\newglossaryentry{gl:test}{
    name={Test},
    description={Test description}
  }
  
%%% define the acronym and use the see= option
\newglossaryentry{ac:test}{
  type=\acronymtype, 
  name={tst}, 
  description={Test, \emph{Glossary:} \gls{gl:test}},
  first={Test (tst)\glsadd{gl:test}}, 
  }

% Acronyms
\newacronym{ac:haw}{HAW}{Hochschule für Angewandte Wissenschaften}
\newacronym{ac:it}{IT}{Information-Technologie}
\newacronym{ac:poc}{PoC}{Proof of Concept}
\newacronym{ac:gui}{GUI}{Graphical User Interface}
\newacronym{ac:ux}{UX}{User Experince}
\newacronym{ac:iam}{IAM}{Identity and Access Management}
\newacronym{ac:csv}{CSV}{Comma Separated Values}
\newacronym{ac:grpc}{gRPC}{google Remote Procedure Call}
\newacronym{ac:api}{API}{Application Programming Interface}
\newacronym{ac:rest}{REST}{REpresentational State Transfer}

% Symbols
\newglossaryentry{sy:ohm}{
  type=symbols,
  name={\ensuremath{\Omega}},
  sort=Ohm, symbol={\ensuremath{\Omega}},
  description={unit of electrical resistance}}
